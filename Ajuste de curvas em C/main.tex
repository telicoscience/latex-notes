\documentclass{article}

% Language setting
% Replace `english' with e.g. `spanish' to change the document language
\usepackage[portuguese]{babel}

% Set page size and margins
% Replace `letterpaper' with`a4paper' for UK/EU standard size
\usepackage[letterpaper,top=2cm,bottom=2cm,left=3cm,right=3cm,marginparwidth=1.75cm]{geometry}

% Useful packages
\usepackage{amsmath}
\usepackage{graphicx}
\usepackage{listings}



\usepackage[colorlinks=true, allcolors=blue]{hyperref}
\newtheorem{theorem}{Teorema}[section]
\newtheorem{proof}{Demonstração}[section]

\title{Ajuste de curvas}
\author{Télico Oliveira}

\begin{document}
\maketitle
\section{Resumo}
A partir de um problema disponível no livro Cálculo numérico e aplicações, vamos discorrer sobre ajustes de curvas e implementar um algoritmo que calcula a reta que mais se aproxima de um conjunto de pontos dados utilizando o método dos mínimos quadrados. 
\\
\textbf{Problema} Construir um diagrama de dispersão das medidas das variáveis x e y. 
\begin{eqnarray*}
    x_i = \{1.3, 3.4, 5.1, 6.8, 8.0 \} \\
    y_i = \{2.0, 5.2, 3.8, 6.1, 5.8 \}
\end{eqnarray*}
\begin{equation}
    d_i = y_i - \hat{y_i}
\end{equation}
\begin{equation}
    D = \sum_{i = 1}^n d_i^2
\end{equation}
A variável D calcula os desvios das ordenadas de cada ponto em relação a uma determinada reta e faz o somatório dos quadrados desses desvios. 

O problema consiste em encontrar os coeficientes $b_0$ e $b_1$ dessa reta. 
\begin{equation}
    \hat{y_i} = b_0 + b_1 x_i
\end{equation}
Considere que D é uma função dos coeficientes $b_0$ e $b_1$ $D(b_0, b_1)$, ou seja
\begin{equation*}
    D(b_0, b_1) = \sum_{i = 1}^n d_i^2
\end{equation*}
\begin{equation*}
    D(b_0, b_1) = \sum_{i = 1}^n \left[ y_i - (b_0 + b_1x_i)\right]^2
\end{equation*}
\section{Escolha da melhor reta}

A reta que mais se aproxima do conjunto de pontos dados será aquela que possuir o menor desvio . para encontrarmos esta reta, precisamos calcular as derivadas parciais de $D(b_0, b_1)$ e igualá-las a zero. 
\begin{equation}
     D(b_0, b_1) = \sum_{i = 1}^n (y_i - b_0 + b_1x)^2
\end{equation}
\begin{equation}
    \frac{\partial D}{\partial b_0} = -2 \sum_{i = 1}^n(y_i b_0 -b_1x_i)
\end{equation}

\begin{equation}
    \frac{\partial D}{\partial b_1} = -2 \sum_{i = 1}^n(y_i b_0 -b_1x_1)x_i
\end{equation}
Igualando as derivadas parciais a zero, vem: 
\begin{equation*}
  -2 \sum_{i = 1}^n(y_i b_0 -b_1x_i) = 0
\end{equation*}
\begin{equation*}
 -2 \sum_{i = 1}^n(y_i b_0 -b_1x_i) = 0
\end{equation*}
\begin{eqnarray}
\begin{cases}
\sum y_i - \sum b_0 - \sum b_1 x_i = 0 \\
\sum x_i y_i - \sum b_0 x_i - \sum b_1 x_i^2 = 0 
\end{cases}
\end{eqnarray}
\begin{eqnarray}
\begin{cases}
 (n) b_0 + (\sum x_i) b_1 = \sum y_i  \\
 (\sum  x_i)b_0 + (\sum x_i^2)b_1  = \sum x_i y_i 
\end{cases}
\end{eqnarray}
Resolvendo o sistema para $b_0$ e $b_1$ por qualquer um dos métodos, obtemos como resultado 
\begin{eqnarray}
b_1 = \frac{n \cdot \sum x_i y_i - \sum x_i \sum y_i \sum y_i}{n\cdot \sum x_i^2 - (\sum x_i)^2} \\ 
b_0 = \frac{\sum y_i - (\sum x_i)b_1}{n}
\end{eqnarray}
\section{Implementação do ajuste da melhor curva em C}

\begin{lstlisting} 

#include "stdio.h"
#include "stdlib.h"
#include "math.h"

int main(){
 double X[5] ={1.3, 3.4, 5.1, 6.8, 8.0} ; 
 double Y[5] = {2.0, 5.2, 3.8, 6.1, 5.8}; 
 double reta1[5]; 
 double reta2[5]; 
 double d1[5];
 double d2[5];  
 double D1 = 0;
 double D2 = 0;  
 double sx, sy, sxy, sx2, b0, b1; 
int i; 
int n = 5; 
	printf("\ni  X[i], Y[i]"); 
 for (i = 0; i < n; i ++ ){
 	reta1[i] = 0 + 1* X[i];
 	d1[i] = Y[i]- reta1[i];  
 	reta2[i] = 4.5 + 0*X[i];
 	d2[i] = Y[i]- reta2[i];  


 	printf("\n %i,  %.1f, %.1f",i + 1,  X[i], Y[i]); 
	printf("\n"); 
	
 }
 for (i = 0; i < n; i ++ ){
 	D1 += pow(d1[i], 2);
 	D2 += pow(d2[i], 2); 
 //	printf("\n%.2f, %2.f\n", D1, D2); 
 }
 //Implementa os somatorios 
 for (i = 0; i < n; i++){
 	 sx += X[i]; 
 	 sy +=Y[i]; 
 	 sxy +=X[i]*Y[i];
 	 sx2 +=pow(X[i], 2);   
 	//printf("sx = %.1f  sy = %.1f  sxy = %.1f sx2 = %.1f\n", sx, sy, sxy, sx2); 
 }
//Calcula os coeficientes da reta 
b1 = (n* sxy - sx*sy)/(n*sx2- pow(sx, 2));
b0 = (sy - sx*b1)/n; 
printf("\nY =  %.1f +  %.1fX\n ", b0, b1); 
return 0; 

}
\end{lstlisting}
\end{document}