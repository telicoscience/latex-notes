\documentclass{article}

% Language setting
% Replace `english' with e.g. `spanish' to change the document language
\usepackage[portuguese]{babel}

% Set page size and margins
% Replace `letterpaper' with`a4paper' for UK/EU standard size
\usepackage[letterpaper,top=2cm,bottom=2cm,left=3cm,right=3cm,marginparwidth=1.75cm]{geometry}

% Useful packages
\usepackage{amsmath}
\usepackage{graphicx}
\usepackage[colorlinks=true, allcolors=blue]{hyperref}

\title{Expressões numéricas}
\author{Télico Oliveira}

\begin{document}
\maketitle
O objetivo dessa aula é aprofundar os conhecimentos sobre as operações já estudadas, \textbf{adição, subtração, multiplicação, divisão e potenciação}. \cite{dante}

Para resolvermos problemas envolvendo tais operações, é necessário seguir uma ordem tanto para as operações quanto para os símbolos que aparecem para organizar as expressões, no caso, os parênteses (), colchetes [] e chaves \{\}. Veja como se dá essa ordem.
\begin{itemize}
    \item 1º) Resolvemos as expressões contidas dentro dos parênteses (). 
    \item 2°) Resolvemos as expressões de dentro dos colchetes []. 
    \item 3º) Por último, resolvemos as operações dentro das chaves \{\}.
\end{itemize}
Com relação à ordem das operações, precisamos seguir a seguinte regra: 
\begin{itemize}
    \item 1º) Potenciação e radiciação. 
    \item 2º) Multiplicação e divisão.
    \item 3º) Adição e subtração. 
\end{itemize}
\section{Exemplos}

\begin{enumerate}
    \item $2\times (10 + 5) \\
     =  2 \cdot 10 \\
     =20.$
    \item  $2 \times 10 + 5\\ 
     = 20 + 5 \\
      = 25.$
    \item  $15 : (5 - 2)\\ 
    = 15: 3 \\
     = 5.$
    \item  $(4 + 6)^2 \\
     = 10^2 \\
      = 100.$
    \item $ (10 + 4) \times (8 - 6) \\ 
     = 14 \times 2 \\
     = 28. $
    \item $(4 + 18) \times 2 \\
     = 22 \times 2 \\ 
      = 44. $
    \item $\{20 - [6 + (4+1)\times 2] + 1\}\times 3 \\
     = \{20 - [6 + 5\times 2] + 1\}\times 3 \\
     = \{20 - [6 + 10] + 1\}\times 3\\
     = \{20 - 16 + 1\}\times 3\\
     = 5 \times 3\\
      = 15.$
     
    \item $\{5^2 -  [(5+ 3) \times 2] - 1^2\}\times 5
     = \{5^2 -  [8 \times 2] - 1^2\}\times 5\\
      = \{5^2 -  16 - 1^2\}\times 5\\
       = \{25 -  16 - 1 \}\times 5\\
        =8\times 5  \\
         = 40.$
\end{enumerate}
O resultado de uma expressão numérica é único. Se a ordem correta das operações ou dos símbolos não for respeitada, chegaremos em resultados errados. 
\bibliographystyle{alpha}
\bibliography{sample}
\end{document}